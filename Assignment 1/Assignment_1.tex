\documentclass[12pt,a4paper]{article} 
\renewcommand{\baselinestretch}{1.5}
\usepackage[utf8]{inputenc}
\usepackage[letterspace=2]{microtype}
\usepackage{enumitem}
\begin{document}

%-------------------------------
    %	TITLE SECTION
%-------------------------------
    
    \hrule \medskip % Upper rule
    \begin{minipage}{0.3\textwidth}
    \raggedright
    \small
    \textbf{Suraj Kumar Sahu}
    \\
    19111060
    \\
    BIOMED 5th SEM
    
    \end{minipage}
    \begin{minipage}{0.48\textwidth} 
    \centering 
    \large 
    \textbf{ASSIGNMENT 1}\\
    \footnotesize
    \textbf{Philosophy Of Artificial Intelligence}\\ 
    \end{minipage}
    \begin{minipage}{0.18\textwidth}
    \raggedleft
    20 July 2021 \hfill\\
    \end{minipage}
    \medskip\hrule 
    \bigskip

\subsection*{Summary}
Artificial intelligence philosophy is a subset of technology philosophy that investigates artificial intelligence and its implications for knowledge and understanding of intelligence, ethics, consciousness, epistemology, and free will.
\\
\\
  \textbf{Important propositions in the philosophy of AI include some of the following:}\\
    --Turing's "polite convention": If a machine behaves as intelligently as a human being, then it is as intelligent as a human being.\\
    --The Dartmouth proposal: "Every aspect of learning or any other feature of intelligence can be so precisely described that a machine can be made to simulate it."\\
    --Allen Newell and Herbert A. Simon's physical symbol system hypothesis: "A physical symbol system has the necessary and sufficient means of general intelligent action."\\

\subsubsection* {Can a machine display general intelligence?}
Can a machine be built that will solve all of the problems human intelligence encounters? This question determines the future scope and direction of Advanced research. Only the behaviour and the problems of psychology, scientific interest and philosophers are involved; the question is whether an engine actually thinking (as a person thinks), or does what it thinks. It is important to answer that question.
\subsubsection* {Can a machine have a mind, consciousness, and mental states?}
The question is about the position that John Searle defines as "Strong AI": a physical symbolic system can have a mental state and a state of mind.
Searle distinguished this point from what was known as 'weak AI': it is possible to operate an intelligent physical symbol system.
\\

\textbf{Important keywords :- }
\begin{itemize}[itemsep=10pt]
\item Algorithm
\item Artificial intelligence
\item Cognitive computing:
\item Consciousness
\item Computation
\item Deep learning
\item Ethics
\item Intelligent agents
\item Machine Learning
\item Unsupervised learning: 
\end{itemize}
\end{document}
