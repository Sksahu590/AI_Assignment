\documentclass[12pt,a4paper]{article} 
\renewcommand{\baselinestretch}{1.5}
\usepackage[utf8]{inputenc}
\usepackage[letterspace=2]{microtype}

\begin{document}

%-------------------------------
    %	TITLE SECTION
%-------------------------------
    
    \hrule \medskip % Upper rule
    \begin{minipage}{0.3\textwidth}
    \raggedright
    \small
    \textbf{Suraj Kumar Sahu}
    \\
    19111060
    \\
    BIOMED 5th SEM
    
    \end{minipage}
    \begin{minipage}{0.45\textwidth} 
    \centering 
    \large 
    \textbf{ASSIGNMENT 3}\\
    \normalsize 
    \textbf{Moravec’s Paradox}\\ 
    \end{minipage}
    \begin{minipage}{0.2\textwidth}
    \raggedleft
    \today\hfill\\
    \end{minipage}
    \medskip\hrule 
    \bigskip

\subsection*{Introduction}
Moravec's paradox is the observation that, contrary to popular belief, reasoning requires very little computation while sensorimotor skills require enormous computational resources. Hans Moravec, Rodney Brooks, Marvin Minsky, and others articulated the principle. It is relatively simple to make computers perform at adult levels on intelligence tests or when playing checkers, but it is difficult or impossible to give them the perception skills of a one-year-old.

\subsubsection*{The biological basis of human skills}
All human abilities are implemented biologically, using machinery Natural selection has had more time to improve the design of a skill as it has aged. He claims that while skills that appear effortless are difficult to reverse-engineer, skills that require effort may not be difficult to engineer at all.

\subsubsection*{Historical influence on artificial intelligence}
Leading researchers in the early days of artificial intelligence research frequently predicted that they would be able to create thinking machines in a matter of decades. They'd had success writing programmes that used logic, solved algebra and geometry problems, and played games like checkers and chess. Rodney Brooks made the decision to create intelligent machines with no cognition. It's just sensing and acting. This new direction, which he dubbed Nouvelle AI, had a significant impact on robotics research and AI.

\end{document}
