\documentclass[12pt,a4paper]{article} 
\renewcommand{\baselinestretch}{1.5}
\usepackage[utf8]{inputenc}
\usepackage[letterspace=2]{microtype}
\usepackage{graphicx}

\begin{document}

%-------------------------------
    %	TITLE SECTION
%-------------------------------
    
    \hrule \medskip % Upper rule
    \begin{minipage}{0.3\textwidth}
    \raggedright
    \small
    \textbf{Suraj Kumar Sahu}
    \\
    19111060
    \\
    BIOMED 5th SEM
    
    \end{minipage}
    \begin{minipage}{0.45\textwidth} 
    \centering 
    \large 
    \textbf{ASSIGNMENT 4}\\
    \normalsize 
    \textbf{The Bermuda Triangle}\\ 
    \end{minipage}
    \begin{minipage}{0.2\textwidth}
    \raggedleft
    \today\hfill\\
    \end{minipage}
    \medskip\hrule 
    \bigskip

\subsection*{Introduction}
The Bermuda Triangle, also known as the Devil's Triangle, is a stretch of the Atlantic Ocean bordered by a line running from Florida to the Bermuda Islands. It is one of the greatest mysteries of our time, as several ships and planes appear to mysteriously vanish over that area. 
\\
\\
\begin{figure}[htp]
    \centering
    \includegraphics[width=.8\textwidth]{img.png}
    \caption{The Bermuda Triangle}
    \label{fig:galaxy}
\end{figure}

\subsubsection*{Legends of Bermuda}
The stories of Flight 19, a group of five US torpedo bombers that disappeared in the triangle in 1945, are part of the Bermuda triangle legends. A rescue aircraft was also missing to find them.
\\
Altogether, as far as we know, \textbf{75+ planes and hundreds of ships met their demise in the Bermuda Triangle}. Possible causes for the catastrophes have been proposed over time, ranging from the paranormal, electromagnetic interference that causes compass problems, bad weather, the gulf stream, and large undersea fields of methane.

\subsubsection*{Best Theory so far on basis of AI}
Meteorologists have suggested a fascinating theory that 170 mph air bombs full of wind are responsible for the mysteries pervading the Bermuda Triangle. These air pockets are causing all the malfunctions, sinking and falling aircraft.
\\
\begin{figure}[htp]
    \centering
    \includegraphics[width=.8\textwidth]{img1.png}
    \caption{Science Channel}
    \label{fig:galaxy}
\end{figure}
\\
In studying images of a NASA satellite, the scientists concluded that some of the clouds measure between 20 and 55 miles. Waves can reach up to 45 feet within these monsters of the wind. Moreover, there are direct edges in the clouds.
\end{document}
