\documentclass[12pt,a4paper]{article} 
\renewcommand{\baselinestretch}{1.5}
\usepackage[utf8]{inputenc}
\usepackage[letterspace=2]{microtype}

\begin{document}

%-------------------------------
    %	TITLE SECTION
%-------------------------------
    
    \hrule \medskip % Upper rule
    \begin{minipage}{0.3\textwidth}
    \raggedright
    \small
    \textbf{Suraj Kumar Sahu}
    \\
    19111060
    \\
    BIOMED 5th SEM
    
    \end{minipage}
    \begin{minipage}{0.45\textwidth} 
    \centering 
    \large 
    \textbf{ASSIGNMENT 3}\\
    \normalsize 
    \textbf{Moravec’s Paradox}\\ 
    \end{minipage}
    \begin{minipage}{0.2\textwidth}
    \raggedleft
    20 July 2021 \hfill\\
    \end{minipage}
    \medskip\hrule 
    \bigskip

\subsection*{Introduction} of mathematical logic
Incompleteness of Gödel's theorems are two theorems that deal with the limits of probability in official axiomatic theories. The first inaccuracy theory shows that the arithmetic of natural numbers is not demonstrably supported by a coherent axiom system whose theorems can be listed using an effective process. There are always statements regarding natural numbers true for such a consistently formal system but which are not proven in the system. The second theorem for incompleteness and extension of the first theorem.

\subsubsection*{First Incompleteness Theorem  } 
Any Consistent Formal system F within which a certain amount of elementary arithmetic can be carried out is incomplete i.e. there are statements of the language of F which can neither be proved nor disproved out F.

\subsubsection*{Second Incompleteness Theorem  } 
For each formal system F containing basic arithmetic, it is possible to canonically define a formula Cons(F) expressing the consistency of F. This formula expresses the property that "there does not exist a natural number coding a formal derivation within the system F whose conclusion is a syntactic contradiction." The syntactic contradiction is often taken to be "0=1", in which case Cons(F) states "there is no natural number that codes a derivation of '0=1' from the axioms of F" .

\end{document}
